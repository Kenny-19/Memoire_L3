 \documentclass{report}

 
\begin{document}
	\author{Tshibangu Ntumba Kenny}
	 
	\begin{Huge}
 Introduction Générale 
	\end{Huge}
	 \begin{description}
	 	\item[ ] Pour Rappel mon sujet c'est la :"Mise en Place d'un système de securite dans un Serveur WEB"
	 \end{description}
 \section{ Quelques petites Définition}
 \subsection{ La Securite Informatique  }
 
 \paragraph{ }  La sécurité informatique est l'ensemble des mesures techniques, organisationnelles et juridiques mises en place pour protéger les systèmes informatiques, les réseaux et les données contre les attaques, les pertes ou les altérations. Elle vise à garantir la confidentialité, l'intégrité et la disponibilité des informations stockées sur les systèmes informatiques, ainsi que la protection de la vie privée et des droits de propriété intellectuelle.
 
 \paragraph{ } La sécurité informatique englobe un large éventail de domaines, tels que la sécurité des réseaux, la sécurité des systèmes d'exploitation, la sécurité des applications, la sécurité des données, la sécurité physique, la gestion des identités et des accès, la conformité aux normes de sécurité, la surveillance et la détection des incidents de sécurité, ainsi que la réponse aux incidents de sécurité.
 
 \paragraph{ }  La sécurité informatique est devenue un enjeu majeur dans le monde numérique d’aujourd’hui, où les attaques informatiques sont de plus en plus sophistiquées et fréquentes. La mise en place d'une politique de sécurité informatique efficace est donc essentielle pour protéger les systèmes informatiques et les données sensibles contre les menaces potentielles et assurer la continuité des activités des organisations qui les utilisent.
 
  \subsection{ La Cyber Sécurité }
  \paragraph{ }
  La cybersécurité, également appelée sécurité informatique ou sécurité des technologies de l'information, est l'ensemble des mesures techniques, organisationnelles et juridiques mises en place pour protéger les systèmes informatiques, les réseaux et les données contre les attaques, les pertes ou les altérations. Elle vise à garantir la confidentialité, l'intégrité et la disponibilité des informations stockées sur les systèmes informatiques, ainsi que la protection de la vie privée et des droits de propriété intellectuelle.
  \paragraph{ }
  La cybersécurité englobe un large éventail de domaines, tels que la sécurité des réseaux, la sécurité des systèmes d'exploitation, la sécurité des applications, la sécurité des données, la sécurité physique, la gestion des identités et des accès, la conformité aux normes de sécurité, la surveillance et la détection des incidents de sécurité, ainsi que la réponse aux incidents de sécurité.
  \paragraph{ }
  La cybersécurité est devenue un enjeu majeur dans le monde numérique d'aujourd'hui, où les attaques informatiques sont de plus en plus sophistiquées et fréquentes. Elle est essentielle pour protéger les systèmes informatiques et les données sensibles contre les menaces potentielles et assurer la continuité des activités des organisations qui les utilisent. La cybersécurité est également importante pour protéger les utilisateurs finaux, tels que les consommateurs et les employés, contre les risques de vol d'identité, de fraude en ligne et d'autres formes de cybercriminalité.
   
\end{document}