
	\author{Tshibangu Ntumba Kenny}
	
	% page de garde 
	
	
	\thispagestyle{empty}
	
	\begin{center}
		\textbf{\large MINISTÈRE DE L'ENSEIGNEMENT SUPÉRIEUR ET UNIVERSITAIRE}
	\end{center}
	
	% logo
	\begin{center}
		\includegraphics[width=0.7\textwidth]{PhotoMemoire/Logo.jpg}
	\end{center}
	
	\begin{center}
		\textbf{\large FACULTÉ DES SCIENCES INFORMATIQUES}
	\end{center}
	
	% title
	\vspace{2cm}
	\begin{center}
		\hrulefill  \\
		\large{
			\textbf{
			  MISE EN PLACE D'UN SYSTÈME DE SÉCURITÉ DANS UN SERVEUR WEB
			}
		} \\
		\hrulefill
	\end{center}
	
	\vspace{0.5cm}
	% informations
	\hfill
	\begin{tabular}{lr}
		Présenté par: & \textbf{Tshibangu Ntumba Kenny} \\
		Option: & \textbf{RÉSEAUX ET INFRASTRUCTURES} \\
	\end{tabular}
	
	\vspace{0.5cm}
	\hfill
	\begin{tabular}{p{9.5cm}}
		Travail de fin de cycle en vue de l'obtention du diplôme de licence en Sciences Informatiques
	\end{tabular}
	
	% année académique 
	\vfill
	\begin{center}
		\large{
			\textbf{
			Juillet - 2023
			}
		}
	\end{center}
	\newpage
	\thispagestyle{empty}
\begin{center}
	\textbf{\large MINISTÈRE DE L'ENSEIGNEMENT SUPÉRIEUR ET UNIVERSITAIRE}
\end{center}

% logo
\begin{center}
	\includegraphics[scale=0.15]{PhotoMemoire/Logo.jpg}
\end{center}

\begin{center}
	\textbf{\large FACULTÉ DES SCIENCES INFORMATIQUES}
\end{center}


% title 
\vspace{3cm}
\begin{center}
	\hrulefill \\
	\large{
		\textbf{
			 MISE EN PLACE D'UN SYSTÈME DE SÉCURITÉ DANS UN SERVEUR WEB	
		}
	} \\
	\hrulefill
\end{center}


% informations
\vspace{0.5cm}
\hfill
\begin{tabular}{lr}
	Présenté par: & \textbf{Tshibangu Ntumba Kenny} \\
	Directeur: & \textbf{Prof.Ebedon Mufind} \\
	Encadreur: & \textbf{Ass. Robert Makila} \\
	Option: & \textbf{RÉSEAUX ET INFRASTRUCTURES}
\end{tabular}

\vspace{0.5cm}
\hfill
\begin{tabular}{p{9.5cm}}
	Travail de fin de cycle en vue de l'obtention du diplôme de licence en Sciences Informatiques
\end{tabular}

% année académique 
\vfill
\begin{center}
	\large{
		\textbf{
			ANNÉE ACADÉMIQUE : 2022-2023
		}
	}
\end{center}
 \begin{abstract}
	\begin{LARGE}
		Résumé
	\end{LARGE}
	\vspace{2cm}
	\begin{center}
		
		La cybersécurité est un domaine crucial dans le monde de l'informatique, car elle concerne la protection des systèmes informatiques, des réseaux et des données contre les menaces numériques telles que les attaques de pirates informatiques, les virus et les logiciels malveillants.
		Dans le contexte des serveurs, la sécurité est essentielle car les serveurs sont souvent utilisés pour stocker des données critiques et des informations sensibles.\\
		Pour garantir la sécurité des serveurs, il est important de mettre en place des protocoles de sécurité appropriés, tels que l'authentification forte, le chiffrement des données et la surveillance en temps réel.\\
		La sécurité informatique implique également la sensibilisation à la sécurité, la formation des utilisateurs et la mise en place de politiques de sécurité strictes pour garantir la sécurité des informations. \\
		Pour un travail de fin de cycle   sur ce sujet, il est important de se concentrer sur les dernières tendances et technologies en matière de cybersécurité et de sécurité des serveurs, en explorant les différentes méthodes et techniques utilisées pour protéger les systèmes informatiques.
		\vfill
		
	\end{center}
	\textbf{Mots Clés} -- Cybersecurite,Pirates Informatique,Serveur,Système Informatique .
\end{abstract}
\begin{abstract}
	\begin{LARGE}	Épigraphe \end{LARGE}
	\vfill
	\begin{center}
		\textbf{" La Cybercriminalité est une menace pour toutes les entreprises et institutions , grandes et petites , et nous devons tous travailler ensemble pour nous protéger"}
	\end{center}
	\vfill
	--\textbf{Chris Hadnagy}
\end{abstract}
\begin{center}
	\chapter*{Dédicace}
	\begin{center}
		Nous tenons à dédier ce  travail à notre  très cher père  \textbf{Jean-Pierre Ntumba Tshibangu}  et à notre douce et tendre mère \textbf{Aimerance Binyngela  Ngalula } , qui  à eux deux ont été nos plus grands soutiens tout au long de notre vie.\\
		A  notre Famille , nos sœurs et nos frères   de \textbf{Keren à  Karim } de \textbf{Giovanni a Christiano} de \textbf{Soraya à Yohan} de \textbf{Allan à Stacey}  de \textbf{Amelia à Talia } de \textbf{Caris à Nathan}... nous ne saurions tous vous citer .\\
		
		A nos grands parents \textbf{Felicien Tshilolo} et \textbf{Veronique Tshiya} qui ont été une source de sagesse .\\
		
		Nous  vous remercions de votre soutien et de votre amour inconditionnel , vous avez toute notre gratitude .\\ 
		
		Votre amour, vos encouragement et votre patience nous ont permis de poursuivre nos rêves et d'atteindre nos objectifs.\\
		Nous  sommes particulièrement reconnaissant pour votre soutien lors de nos études, où vous avez  toujours été là pour nous  encourager à donner le meilleur de nous-même et à nous rappeler l'importance de la persévérance.\\
		Sans votre soutien inconditionnel, nous ne serions pas là où nous sommes aujourd'hui.\\
		Nous  sommes fier de vous montrer les fruits de nos efforts et nous espérons que notre  travail vous apportera autant de joie et de fierté qu'ils  nous ont apporté tout au long de ma vie.\\
	\end{center}
	
\end{center}
\begin{center}
	\chapter*{Remerciements}
	
	Nous tenons à exprimer notre profonde reconnaissance envers notre directeur de mémoire Le Professeur \textbf{Ebedon Mufind} pour nous avoir dirigé d'une main de maitre dans le cadre de l'élaboration de ce travail ,ses conseils  nous ont apporter beaucoup de connaissance dans le domaine de l'informatique en Général;\\
	
	A notre  Co-directeur  Monsieur \textbf{Robert Makila} pour ses corrections et ces orientations qui nous ont apportés des connaissances insoupçonnées dans le domaine informatique ,  nous vous remercions chaleureusement .\\
	
	Nous  tenons à egalement a remercier tout les professeurs , intervenants et personnes  qui ont eu a nous  orienter dans nos recherches entre autre \textbf{Monsieur Jonathan Kabemba ,Monsieur Godwill Ilunga ,Monsieur Landry Mbale } ..\\
	
	Nous  tenons à remercier nos amis pour leurs soutiens ,les bons moments passés ensemble ,les moments de rire d'apprentissage et de loisir.\\
	
	A nos collègues \textbf{ Arsène Sefu , Kabamb Ismael, Wilfried Musanzi , Jean-Baptiste Makabu ,Ortega Kabwe ,Jenovic Dinanga , 
		Mutombo Laurent,Lionel Tshitenge, Eliel Luwala, Audry Mutombo, Issacar Mwal,Kevin Milenda ,  Nyembo Myriam ,Elnola Moteade , Eliezer Mbunda ,Audra Marero, Acaica Ilunga, Gabriella Kankolongo } et à nos camarades de  la Promotion de L3 Informatique 2022-2023.\\
	
	Nous tenons à remercier nos amis entre autre \textbf{Hermiel Disanka } ,\textbf{Naomie Kaind} ,\textbf{Emmannuella Tudienu},\textbf{Erwin Kalombo,Kasinda Plamedie,Elie nzongola, Flory Lyonze} 
	
	A vous Tous je vous remercies sincèrement pour le soutient et l'amour que j'ai eu a votre égard. 
	
	
	
	
\end{center}

\tableofcontents
\begin{center}
	\chapter*{Liste Des Acronymes}
	\begin{itemize}
		\item \textbf{Apache} Apache HTTP Server .
		\item \textbf{IIS} Internet Information Services .
		\item \textbf{FTP} File Transfer Protocol .
		\item  \textbf{SMTP} Simple Mail Transfer Protocol  Server.
		\item \textbf{DNS}  Domain Name System  Server.
		\item \textbf{DHCP} Dynamic Host Configuration Protocol  Server.
		\item \textbf{DDoS } Distributed Denial of Service.
		\item \textbf{Dos} Denial of Service.
		\item \textbf{URL} Uniform Resource Locator.
		\item  \textbf{HTTP} Hypertext Transfer Protocol .
		\item \textbf{HTTPS} Secure Hypertext Transfer Protocol .
		\item \textbf{SFTP} Secure File Transfer Protocol  .
		\item  \textbf{IMAP} Internet Message Access Protocol .
		\item \textbf{TCP} Transmission Control Protocol .
		\item  \textbf{XSS} Cross-Site Scipting
		\item  \textbf{CSS} Cascading Style Sheet
		\item \textbf{IP} Internet Protocol
		\item  \textbf{PHP} Hypertext Preprocessor.
		\item \textbf{SQL} Structured Query Language
		\item \textbf{POP} Post Office Control
		\item  \textbf{DEP} Défense EN Profondeur.
		\item \textbf{SOC} Security Operation Center .
		\item \textbf{SIEM} Security Information and Event Manager .
		\item \textbf{CTI} Cyber Threat Intelligence.
		\item \textbf{ISCM} Information Security Continous Monitoring.
		\item \textbf{IoT} Internet Des objets.
		\item \textbf{EDR} Endpoint detection and Response.
		\item \textbf{MDR} Managed Detection Reponse.
		\item \textbf{NGAV} Next Generation Antivirus .
		\item \textbf{SSL} Secure Socket Layer.
		\item \textbf{VPN} Virtual Port Network
		\item \textbf{IA} Intelligence Artificielle.
		
		
	\end{itemize}
	
\end{center}
\listoffigures
\pagebreak 
\listoftables
\pagebreak
		\newpage
	   
	     
	    
	    
	    
	    
    
    \pagenumbering{arabic}
	\begin{Huge}
		\chapter{ Introduction Générale}
	\end{Huge}
 \paragraph{ }
\section{La Cybersécurité}
 est la pratique consistant à protéger les systèmes, les réseaux et les données contre tout accès, utilisation, divulgation, perturbation, modification ou destruction non autorisés.\\
 C'est un domaine critique dans le monde d’aujourd'hui, car notre dépendance à l'égard de la technologie ne cesse de croître.\\
Les serveurs sont des composants essentiels de toute infrastructure de Cybersécurité.\\
 Ils fournissent un emplacement central pour le stockage et le traitement des données et peuvent être utilisés pour héberger des applications et des services utilisés par les employés, les clients et les partenaires.\\
Ce travail fournira une introduction à la Cybersécurité et aux serveurs.\\
 Il couvrira les sujets suivants :
 
\begin{itemize}
	\item[$\bullet$] Les bases de la Cybersécurité.
	\item[$\bullet$] Les différents types de serveurs.
	\item[$\bullet$] Les défis de sécurité associés aux serveurs.
	\item[$\bullet$] Les bonnes pratiques de sécurisation des serveurs.
\end{itemize}

À la fin de ce travail, vous aurez une compréhension de base de la Cybersécurité et des serveurs.\\ Vous serez en mesure d'identifier les risques de sécurité associés aux serveurs et vous serez en mesure de mettre en place les meilleures pratiques pour protéger vos serveurs.
\subsection{Pourquoi la Cybersécurité est-elle importante ?}   
La Cybersécurité est importante car elle protège notre infrastructure critique, nos informations personnelles et nos actifs financiers.\\
 Ces dernières années, un certain nombre de cyberattaques très médiatisées ont causé des dommages importants.\\
 Par exemple, l'attaque du ransomware WannaCry en 2017 a infecté plus de 200 000 ordinateurs dans 150 pays.\\ L'attaque a causé des milliards de dollars de dégâts et perturbé des services essentiels tels que les hôpitaux et les écoles.\\
 \section{Qu'est-ce qu'un serveur ? }
Un serveur est un ordinateur qui fournit des ressources à d'autres ordinateurs sur un réseau. \\Les serveurs peuvent être utilisés pour stocker des données, exécuter des applications et fournir des services tels que la messagerie électronique, le partage de fichiers et l'impression.
\subsection{Les défis de sécurité associés aux serveurs}
Les serveurs sont une cible courante pour les cyberattaques car ils contiennent des données précieuses et sont souvent connectés à Internet.\\ Certains des défis de sécurité associés aux serveurs incluent :\\
\begin{itemize}
 \item [$\bullet$] Malware : un malware est un logiciel conçu pour endommager un système informatique. Les logiciels malveillants peuvent être utilisés pour voler des données, installer des portes dérobées ou perturber les opérations.
\item [$\bullet$] Phishing :  Le phishing est un type d'attaque d'ingénierie sociale qui est utilisé pour inciter les utilisateurs à révéler leurs informations personnelles, telles que des mots de passe ou des numéros de carte de crédit.
\item [$\bullet$] Attaques par déni de service  :  les attaques DoS sont conçues pour submerger un serveur de trafic, le rendant indisponible pour les utilisateurs légitimes.
\item [$\bullet$] Violations de données :  Les violations de données sont des incidents au cours desquels des données sensibles sont volées dans un système informatique.
\end{itemize}
 Les violations de données peuvent avoir un impact significatif sur les entreprises, car elles peuvent entraîner des pertes financières, la perte de clients et une atteinte à la réputation.
 \subsection{Les meilleures pratiques pour sécuriser les serveurs :}
Il existe un certain nombre de meilleures pratiques qui peuvent être utilisées pour sécuriser les serveurs. Certaines des meilleures pratiques les plus importantes incluent :\\
\begin{itemize}
\item[$\bullet$] Maintenez vos logiciels à jour :   les mises à jour logicielles incluent souvent des correctifs de sécurité qui peuvent aider à protéger votre serveur contre les vulnérabilités connues.
\item[$\bullet$]Utilisez des mots de passe forts :  Les mots de passe doivent comporter au moins 12 caractères et doivent inclure un mélange de lettres majuscules et minuscules, de chiffres et de symboles.
\item[$\bullet$]Activer l'authentification à deux facteurs :  L'authentification à deux facteurs ajoute une couche de sécurité supplémentaire en demandant aux utilisateurs de saisir un code sur leur téléphone en plus de leur mot de passe.
\item[$\bullet$]Utilisez un pare-feu :  Un pare-feu peut aider à protéger votre serveur contre tout accès non autorisé.
\end{itemize}
 Sauvegardez vos données régulièrement :  Des sauvegardes régulières peuvent aider à minimiser les dommages causés par une violation de données.\\
 \textbf{	En suivant ces meilleures pratiques, vous pouvez aider à protéger vos serveurs contre les cyberattaques.}
   \section{Contexte de recherche}
    Ce travail s'appuie sur les différents domaines   de la  Cybersécurité qui actuellement est déjà devenu un  des points  plus   important dans le domaine de l'informatique 
   Une Solution simple et pratique peut être proposée ;\\
   Pour rendre le serveur Plus sur et plus sécurisé
   \section{Solution technique} 
   La solution la plus proche et la moins vorace en ressource serait de mettre en place un système de sécurité qui respecte le nécessaire des normes de sécurités dans un serveur web actuel.
   \section{Problématique}
   Sur un Serveur web il existe plusieurs sortes de vulnérabilités par lesquels on peut facilement y accéder Comme : 
   \begin{itemize}
   	 \item[$\bullet$]\textbf{ L'injection SQL : }
   	 en bref c'est une attaque qui consiste a insérer du code SQL malveillant  dans les entrées   d'un formulaires ...\\
   	 \item[$\bullet$]\textbf{ Cross-Site Scripting(XSS) :} Comme son nom l'indique c'est une faille de sécurité qui permet un attaquant d'injecter du code malveillant dans une page web ou de faire une redirection vers un site frauduleux 
   \end{itemize}
     
 
   \textbf{Et j'en passe ;\\}
   
  
   \pagebreak
   \begin{itemize}
   	 \item[ ] Voici les  quelques questions que nous allons nous poser tout au long de ce travail :
   \item[\space\textendash]Quel est le descriptif d'un bon système de sécurité ?
   \item[\space\textendash] Quels sont les systèmes de sécurités  que nous allons utiliser ?
   \item[\space\textendash] Sur quel type de serveur web ce système sera efficace ?
   \end{itemize}
   
   \section{Méthodologie}
   Pour arriver a une solution plausible  nous allons utiliser une procédure  qui va nous permettre d'installer des machines virtuelles ; différentes machines virtuelles configurée de différentes façon  ;  et tester différentes approches de sécurisation de serveurs et même essayer de les combinées pour voir si le résultat est solide.
  \section{ Techniques de Recherche }
  Pour ce travail les technique appropriée sera :
  \subsection{La Recherche sur Internet :}
  Parcourir les différents  sites et forums qui proposent des travaux similaires aux miens , des vidéos et tutoriels pour les différentes configurations a faire  pour ce travail ...
  \subsection{La Recherche Documentaire :}
    Utiliser les différents livres,revues ,archives ;
    Qui , en les utilisant pourront m'aider a atteindre mon but , ma solution solide .
 \subsection{La Recherche Expérimentale :}
   Faire des petites expérimentation sur mes machines virtuelles configurées comme des serveurs web.
 \section{Limitation} Dans ce travail nous allons nous limiter a utiliser  le Serveur Apache.\\\
 En raison de sa flexibilité , Apache est largement utilisé  par de nombreuse entreprises.
 Il est Open Source,Multi-Plateforme... 
  \section{Objectif}
 L'objectif visé  dans ce travail est de pouvoir mettre en place un système de sécurité capable de remplir le travail nécessaire qui est actuellement demandé  sur les serveurs web ;\\ 
   De configurer un serveur web assez performant pour remplir les pré-requis    nécessaire qui sont demandés par la communauté  des développeurs   Web qui sont majoritairement amenés a utiliser les Serveurs Web Pour héberger leurs sites internet  .

 
   
   

   
   
     
 