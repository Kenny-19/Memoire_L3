

	\author{Tshibangu Ntumba Kenny}
	
	\begin{titlepage}
		
		 \centering
		
		\textbf{\Large RÉPUBLIQUE DÉMOCRATIQUE DU CONGO
			MINISTÈRE DE L’ENSEIGNEMENT SUPÉRIEUR ET UNIVERSITAIRE}
		\vspace{1.5cm}
		 
		 \includegraphics[width=0.5\textwidth]{PhotoMemoire/Logo.jpg}
		 \vspace{3.5cm}
		 
		 \hrule
		 \vspace{0.6cm}
		 \textbf{\Large "  MISE EN PLACE D'UN SYSTÈME DE SECURITE DANS UN SERVEUR WEB "}
		 \vspace{0.6cm}
		 \hrule
		 \vspace{2.5cm}
		 
		  \large Par :\textbf{Tshibangu Ntumba Kenny} 
		  
		  \paragraph{ } Travail de fin d’études présenté en vue de l’obtention
		  du grade de : licencié en Sciences Informatiques
		\paragraph{ } \Large{ Option : Réseau et Infrastructure}
		 \vfill
		 {\LARGE Année Académique 2022-2023}
		 \centering
	\end{titlepage}
\begin{center}
\tableofcontents
\pagebreak
\end{center}
\begin{center}
	\listoffigures
	\pagebreak 
\end{center}
	    \begin{abstract}
	    	 Résume
	    \end{abstract}
    
	\begin{Huge}
		\chapter{ Introduction Générale}
 
	\end{Huge}
 
 \paragraph{ }
\section{La cybersécurité }
 est la pratique consistant à protéger les systèmes, les réseaux et les données contre tout accès, utilisation, divulgation, perturbation, modification ou destruction non autorisés.\\
 C'est un domaine critique dans le monde d’aujourd’hui, car notre dépendance à l'égard de la technologie ne cesse de croître.\\

Les serveurs sont des composants essentiels de toute infrastructure de cybersécurité.\\ Ils fournissent un emplacement central pour le stockage et le traitement des données et peuvent être utilisés pour héberger des applications et des services utilisés par les employés, les clients et les partenaires.\\

Ce travail fournira une introduction à la cybersécurité et aux serveurs.\\
 Il couvrira les sujets suivants :\\

$\bullet$ Les bases de la cybersécurité\\
$\bullet$ Les différents types de serveurs\\
$\bullet$ Les défis de sécurité associés aux serveurs\\
$\bullet$ Les bonnes pratiques de sécurisation des serveurs\\

À la fin de ce travail, vous aurez une compréhension de base de la cybersécurité et des serveurs.\\ Vous serez en mesure d'identifier les risques de sécurité associés aux serveurs et vous serez en mesure de mettre en place les meilleures pratiques pour protéger vos serveurs.\\

 \paragraph{ } \subsection{ Pourquoi la cybersécurité est-elle importante ? }   

La cybersécurité est importante car elle protège notre infrastructure critique, nos informations personnelles et nos actifs financiers. Ces dernières années, un certain nombre de cyberattaques très médiatisées ont causé des dommages importants.\\
 Par exemple, l'attaque du ransomware WannaCry en 2017 a infecté plus de 200 000 ordinateurs dans 150 pays.\\ L'attaque a causé des milliards de dollars de dégâts et perturbé des services essentiels tels que les hôpitaux et les écoles.\\

  \paragraph{ }
 \section{ Qu'est-ce qu'un serveur ? }

Un serveur est un ordinateur qui fournit des ressources à d'autres ordinateurs sur un réseau. \\Les serveurs peuvent être utilisés pour stocker des données, exécuter des applications et fournir des services tels que la messagerie électronique, le partage de fichiers et l'impression.\\

\subsection{ Les défis de sécurité associés aux serveurs}

Les serveurs sont une cible courante pour les cyberattaques car ils contiennent des données précieuses et sont souvent connectés à Internet.\\ Certains des défis de sécurité associés aux serveurs incluent :

 \paragraph{ } $\bullet$ Malware : un malware est un logiciel conçu pour endommager un système informatique. Les logiciels malveillants peuvent être utilisés pour voler des données, installer des portes dérobées ou perturber les opérations.
  \paragraph{ } $\bullet$Phishing :  Le phishing est un type d'attaque d'ingénierie sociale qui est utilisé pour inciter les utilisateurs à révéler leurs informations personnelles, telles que des mots de passe ou des numéros de carte de crédit.
  
 \paragraph{ }$\bullet$ Attaques par déni de service (DoS) :  les attaques DoS sont conçues pour submerger un serveur de trafic, le rendant indisponible pour les utilisateurs légitimes.\\
 
 \paragraph{ }$\bullet$Violations de données :  Les violations de données sont des incidents au cours desquels des données sensibles sont volées dans un système informatique.
 \\ Les violations de données peuvent avoir un impact significatif sur les entreprises, car elles peuvent entraîner des pertes financières, la perte de clients et une atteinte à la réputation.

 \paragraph{}\subsection{Les meilleures pratiques pour sécuriser les serveurs :}
 

Il existe un certain nombre de meilleures pratiques qui peuvent être utilisées pour sécuriser les serveurs. Certaines des meilleures pratiques les plus importantes incluent :

\paragraph{ } $\bullet$Maintenez vos logiciels à jour :   les mises à jour logicielles incluent souvent des correctifs de sécurité qui peuvent aider à protéger votre serveur contre les vulnérabilités connues.\\
 \paragraph{ }$\bullet$Utilisez des mots de passe forts :  Les mots de passe doivent comporter au moins 12 caractères et doivent inclure un mélange de lettres majuscules et minuscules, de chiffres et de symboles.\\
\paragraph{ }$\bullet$Activer l'authentification à deux facteurs :  L'authentification à deux facteurs ajoute une couche de sécurité supplémentaire en demandant aux utilisateurs de saisir un code sur leur téléphone en plus de leur mot de passe.\\
 \paragraph{ } $\bullet$Utilisez un pare-feu :  Un pare-feu peut aider à protéger votre serveur contre tout accès non autorisé.
 Sauvegardez vos données régulièrement :  Des sauvegardes régulières peuvent aider à minimiser les dommages causés par une violation de données.
\paragraph{ }
  \begin{Large}
  	En suivant ces meilleures pratiques, vous pouvez aider à protéger vos serveurs contre les cyberattaques.
  \end{Large}
  
   
   \section{Contexte de recherche }
   \paragraph{ } Ce travail s'appuie sur les différents domaines   de la  Cybersécurité qui actuellement est déjà devenu un  des points  plus   important dans le domaine de l'informatique 
   Une Solution simple et pratique peut être proposée  \space;
   Pour rendre le serveur Plus sur et plus sécurisé
   \section{Solution technique} 
   \paragraph{ }
   La solution la plus proche et la moins vorace en ressource serait de mettre en place un système de securite qui respecte le nécessaire des normes de securites dans un serveur web actuel.
   \section{Problématique }
   \paragraph{ }
   Sur un Serveur web il existe plusieurs sortes de vulnérabilités par lesquels on peut facilement y accéder Comme : 
     \paragraph {$\bullet$L'injection SQL : }
      en bref c'est une attaque qui consiste a insérer du code SQL malveillant  dans les entrées   d'un formulaires ...
   
   \paragraph{$\bullet$ Cross-Site Scripting(XSS) :} Comme son nom l'indique c'est une faille de securite qui permet un attaquant d'injecter du code malveillant dans une page web ou de faire une redirection vers un site frauduleux 
   \paragraph{ }
   Et j'en passe ;
   \pagebreak
   \paragraph{ }
   Voici les  quelques questions que nous allons nous poser tout au long de ce travail :
   \paragraph{ }
   \textendash \space Quel est le descriptif d'un bon système de securite ?
   \paragraph{ }
   \textendash \space Quels sont les systèmes de securites  que nous allons utiliser ?
   \paragraph{ }
   \textendash \space Sur quel type de serveur web ce système sera efficace ?
   \section{Méthodologie}
   \paragraph{ }
   Pour arriver a une solution plausible  nous allons utiliser une procédure  qui va nous permettre d'installer des machines virtuelles ; différentes machines virtuelles configurée de différentes façon  ;  et tester différentes approches de sécurisation de serveurs et même essayer de les combinées pour voir si le résultat est solide .
  \section{ Techniques }
  \paragraph{ }
  Pour ce travail la technique appropriée sera :
  \subsection{ La Recherche sur Internet :}
  \paragraph{ }
  Parcourir les différents  sites et forums qui proposent des travaux similaires aux miens , des vidéos et tutoriels pour les différentes configurations a faire  pour ce travail ...
    \subsection{La Recherche Documentaire :}
    \paragraph{ }
    Utiliser les différents livres,revues ,archives ;
    Qui , en les utilisant pourront m'aider a atteindre mon but , ma solution solide .
  \subsection{ La Recherche Expérimentale :}
  \paragraph{ }
  Faire des petites expérimentation sur mes machines virtuelles configurées comme des serveurs web  
  \section{Limitation}
  Dans ce travail nous allons nous limiter a utiliser :
  \paragraph{ } (En fonction de l’évolution de mon travail ce point va se remplir ).
  \section{Objectif}
  
  \paragraph{ }L'objectif visé  dans ce travail est de pouvoir mettre en place un système de securite capable de remplir le travail nécessaire qui est actuellement demandé  sur les serveurs web ;
  \paragraph{ } De configurer un serveur web assez performant pour remplir les prérequis    nécessaire qui sont demandés par la communauté  des développeurs   Web qui sont majoritairement amenés a utiliser les Serveurs Web Pour héberger leurs sites internet  .

 
   
   

   
   
     
 